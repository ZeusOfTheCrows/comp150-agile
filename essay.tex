% Please do not change the document class
\documentclass{scrartcl}

% Please do not change these packages
\usepackage[hidelinks]{hyperref}
\usepackage[none]{hyphenat}
\usepackage{setspace}
\doublespace

% You may add additional packages here
\usepackage{amsmath}

% Please include a clear, concise, and descriptive title
\title{How And When To Split User Stories On A Kanban Board}

% Please do not change the subtitle
\subtitle{COMP150 - Agile Development Practice}

% Please put your student number in the author field
\author{1807684}

\begin{document}

\maketitle

\section{Introduction}
\paragraph{}
Agile is a development philosophy that attempts to...%CTRL+F

\section{Sprint Programming}
\paragraph{}
Sprint programming is an important aspect of development with scrum , and almost necessary for the use of kanban boards. It has many positive effects; it breaks a large task—developing an entire game—into shorter and much more manageable chunks, it encourages teams to face problems head on at an early stage allowing them to solve them before they become issues, and it encourages sharing of solutions for problems that may become an issue in the future.

\section{User Stories}
\paragraph{}
User stories are a valuable method of managing a team's task list, as not only does it provide an easy way of managing and prioritising a backlog, as well as clearly showing how much work is done and how mush is left to do; but most importantly, it's understandable even to those in the group that aren't programmers. As they're written from the user's perspective and in a user friendly lexicon, it means that those who don't understand technical "jargon" can still add, remove, and change them. %Also, they 

\section{Story Sizes}
\paragraph{}
Smaller stories are much more motivating; it's much easier to pick up a quick task, complete it, and move on. As tasks are self-selected in a scrum team, this means better efficiency, higher moral, and CTRL+F. Also, as they should be completed quickly, it allows colleagues that can see how long someone has been on a specific story to elect to step in help if they feel someone has been taking too long to complete a particular task\cite{Collab}. %CTRL+F
\paragraph{}
However, they also create a quite a significant mess on the backlog. Having a multitude of tiny, one-hour tasks that have no obvious way of telling what relates to what can cause an unusably cluttered board, negating any positive effects of using the kanban technique.

\section{Splitting Stories}
\paragraph{}
Clinton Keith's book \cite{Keith} advocates starting with large stories, and splitting them at the start of a sprint%CTRL+F
-Conditions of Satisfaction


\section{Conclusion}
\paragraph{}
So as you can see, each have merits. It's definitely worth having epics as they show which tasks are connected much more easily, and they keep the board clean and easily accessible.%CTRL+F

I'm advocating the higher priority epics to be split into smaller stories regularly, long before the start of the sprint they will probably be completed in, and show that they are linked to each other in some way (such as by being the same colour, or by a letter or number in one corner). This allows teams to complete part of an epic during a sprint...%CTRL+F

\section{Placeholders}
\subsection{Introduction}
Write your introduction here. A brief introduction is recommended, which should outline key details of the chosen topic and the reviewed papers, motivate the work, and provide a roadmap of key points to the reader. The motivation is quite important here, as essays should have a contribution (i.e., what is the point of the essay, and what does the reader take away from the essay) and the link between the motivation (in the introduction) and the contribution (in the conclusion) should be made clear.

\subsection{Body}
Write the main body of your essay here. Add more sections if appropriate. You may choose to write about each of your three papers in its own section, or you may choose a different structure. Either way, remember that you are being assessed on technical insight and analysis: it is not enough to merely summarise the contents of the three papers. You must demonstrate the ability to make inferences beyond what is written in the papers, and to draw the three papers together into a single coherent narrative.

Your essay must make a clear recommendation, in terms of which of the three techniques you have reviewed is the best according to whichever metric or metrics you feel is most appropriate. You must justify your choice, backing it up with empirical evidence. However remember that an academic essay is not a murder mystery: you should already have briefly discussed your recommendation in the introduction and in other parts of the essay. Do not save it for a grand reveal at the end.

\subsection{Summary}
Write your conclusion here. The conclusion should do more than summarise the essay, making clear the contribution of the work and highlighting key points, limitations, and outstanding questions. It should not introduce any new content or information.

\bibliographystyle{ieeetr}
\bibliography{references}

\end{document}
